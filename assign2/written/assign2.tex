\documentclass[11pt]{article}

\input{../../tex/defs.tex}

\begin{document}

\hwtitle
  {Assignment 2}
  {Vivek Pandya (vpandya)} %% REPLACE THIS WITH YOUR NAME/ID

\problem{Problem 2}

\textbf{Part 1:}

\begin{mathpar}
\text{Step 1:}\qquad
\ir{D-App-Body}
  {\ir{D-App-Lam}
    {\ir{D-App-Done}
      {\ir{D-Lam}{ \ }{\val{\fun{\_}{x}}}}
      {\dynJ{\{x \rightarrow D\}}{\steps
        {\app{(\fun{x}{\fun{\_}{x}})}{L}}
        {\fun{\_}{x}}}}}
    {\dynJ{\{x \rightarrow D\}}{\steps
      {\app{\app{(\fun{x}{\fun{\_}{x}})}{L}}{*}}
      {\app{(\fun{\_}{x})}{*}}}}}
  {\dynJ{\varnothing}{\steps
    {\app{(\fun{x}{\app{\app{(\fun{x}{\fun{\_}{x}})}{L}}{*}})}{D}}
    {\app{(\fun{x}{\app{(\fun{\_}{x})}{*}})}{D}}}}

 \text{Step 2:}\qquad
  \ir{D-App-Body}
  {\ir{D-App-Body}
  {\ir{D-Var}
  {(x \rightarrow D \in \Gamma )}
  {\dynJ{\{ x \rightarrow D ,  \_ \rightarrow * \}}{\steps{x}{D}}}}
  {\dynJ{\{ x \rightarrow D\}}{\steps{\app{(\fun{\_}{x})}{*}}{\app{(\fun{\_}{D})}{*}}}}}
  {\dynJ{\varnothing}
  {\steps
  {\app{(\fun{x}{\app{(\fun{\_}{x})}{*}})}{D}}
  {\app{(\fun{x}{\app{(\fun{\_}{D})}{*}})}{D}}}}
 
  \\
  \text{Step 3:}\qquad
  \ir{D-App-Body}
  {\ir{D-App-Done}
  {\val{D}}
    {\dynJ{\{x \rightarrow D\}}
  	{\steps
    	{\app{(\fun{\_}{D})}{*}}
    	{D}
    }
  }}
  {\dynJ{\varnothing}
  	{\steps
    	{\app{(\fun{x}{\app{(\fun{\_}{D})}{*}})}{D}}
    	{\app{(\fun{x}{D})}{D}}
    }
  }
  
 \\
 \text{Step 4:}\qquad
 \ir{D-App-Done}{ \val{D} }
 {{\dynJ{\varnothing}{\steps
 		{\app{(\fun{x}{D})}{D}}
 		{D}
 	}
 }
 }
 \end{mathpar}

\textbf{Part 2:}

%Here's examples of how to LaTeX the existing dynamically-scoped lambda calculus rules. Delete them before you submit.
 Following rules in addition to all rules described in problem 2 are required to support let syntax with dynamic scope:
\begin{mathpar}
%\ir{D-Lam}{ \ }{\val{\fun{x}{e}}}

%\ir{D-Var}
%  {x \rightarrow e \in \ctx}
  %{\dynJC{\steps{x}{e}}}

%\ir{D-App-Lam}
  %{\dynJC{\steps{e_\msf{lam}}{e_\msf{lam}'}}}
  %{\dynJC{\steps{\app{e_\msf{lam}}{e_\msf{arg}}}{\app{e_\msf{lam}'}{e_\msf{arg}}}}}

%\ir{D-App-Body}
  %{\dynJ{\ctx, x \rightarrow e_\msf{arg}}{\steps{e_\msf{body}}{e_\msf{body}'}}}
  %{\dynJC{\steps{\app{(\fun{x}{e_\msf{body}})}{e_\msf{arg}}}{\app{(\fun{x}{e_\msf{body}'})}{e_\msf{arg}}}}} \s

%\ir{D-App-Done}
  %{\val{e_\msf{body}}}
  %{\dynJC{\steps{\app{(\fun{x}{e_\msf{body}})}{e_\msf{arg}}}{e_\msf{body}}}}
 
 % TODO: should context alwasy be empty set in the begining ?
 

 
\ir{D-Let1}{\dynJ{\ctx, x \rightarrow e_\msf{var}}{\steps{e_\msf{body}}{e_\msf{body}'}}}
{\dynJC{\steps{let \, x \, = \, e_\msf{var} \, in \, e_\msf{body}}{e_\msf{body}'}}}

\ir{D-Let2}{e_\msf{body} \, \msf{val}}
{\dynJC{\steps{let \, x \, = \, e_\msf{var} \, in \, e_\msf{body}}{e_\msf{body}}}}


\end{mathpar}

\newpage

\problem{Problem 3}

Consider follwing counter example for $\msf{let}$ construct.

$let x : (\steps{\tnum}{\tnum}) = 2\, in\, (x \, 2))$

Now given that  x : $(\steps{\tnum}{\tnum})$ by inversion of T-app rule we can say that $(x\, 2) : \tnum$

So premise for T-let holds here so we can say that

$let x : (\steps{\tnum}{\tnum}) = 2 \, in \, (x\, 2)) : \tnum$

but now we try to step the above expression by applying D-let rule then we get

$e^{`}_\msf{body} = (2\, 2)$ which is a stuck state as we don't have any rule to make a further progress also (2 2) it self is not a val.
Here preservation also don't hold because type of (2 2) is not $\tnum$

Note: This can be fixed if we have restriction on the type of $e_\msf{var}$ , type of $e_\msf{var}$ should be same as                                                                                                                                                                                                                                                                                                                                                                                                                                                                                                                                                                                                                                                                                                                                                                                                                                                                                                                                                     
\tau_\msf{var} .


\begin{mathpar}
%\ir{T-Let}{\typeJ{\ctx,\hasType{x}{\tau_\msf{var}}}{e_\msf{body}}{\tau_\msf{body}}}{\typeJC{(\lett{x}{\tau_\msf{var}}{e_\msf{var}}{e_\msf{body}})}{\tau_\msf{body}}} \\

%\ir{D-Let}{\ }{\steps{\lett{x}{\tau}{e_\msf{var}}{e_\msf{body}}}{\subst{x}{e_\msf{var}}{e_\msf{body}}}}

%\ir{T-Rec}
% {\typeJC{e_\msf{arg}}{\tnum} \\ \typeJC{e_\msf{base}}{\tau} \\
%   \typeJ{\ctx,\hasType{x_\msf{num}}{\tnum},\hasType{x_\msf{acc}}{\tau}}{e_\msf{acc}}{\tau}}
% {\typeJC{\rec{e_\msf{base}}{x_\msf{num}}{x_\msf{acc}}{e_\msf{acc}}{e_\msf{arg}}}{\tau}}

%\ir{D-Rec-Step}
% {\steps{e}{e'}}
%  {\steps
%   {\rec{e_\msf{base}}{x_\msf{num}}{x_\msf{acc}}{e_\msf{acc}}{e}}
%    {\rec{e_\msf{base}}{x_\msf{num}}{x_\msf{acc}}{e_\msf{acc}}{e'}}}

%\ir{D-Rec-Base}
%  {\ }
%  {\steps
%   {\rec{e_\msf{base}}{x_\msf{num}}{x_\msf{acc}}{e_\msf{acc}}{0}}
%   {e_\msf{base}}}

%\ir{D-Rec-Dec}
%  {n >0}
% {\steps
%    {\rec{e_\msf{base}}{x_\msf{num}}{x_\msf{acc}}{e_\msf{acc}}{n}}
%   {[x_\msf{num} \rightarrow n, x_\msf{acc} \rightarrow \rec{e_\msf{base}}{x_\msf{num}}{x_\msf{acc}}{e_\msf{acc}}{n-1}] \ e_\msf{acc}}}
\end{mathpar}
For $\msf{rec}$ construct consider following proofs.
\begin{proof}
Preservation: if$\,  \emptyset \vdash e \, : \tau  \, \text{and} \, \steps{e}{e^\msf{`}} \, \text{then} \, \emptyset \vdash e^\msf{`} : \tau . $\\
Proof. By rule induction on the static semantics.\\

T-Rec: $ \text{if }  \rec{e_\msf{base}}{x_\msf{num}}{x_\msf{acc}}{e_\msf{acc}}{e_\msf{arg}} : \tau and \steps{ \rec{e_\msf{base}}{x_\msf{num}}{x_\msf{acc}}{e_\msf{acc}}{e_\msf{arg}}}{e^\msf{`}} then e^{`} : \tnum \\
\text{ Frist by premises }\\
e_\msf{arg} : \tnum , e_\msf{base} : \tau , \, \,\, \typeJ{\hasType{x_\msf{num}}{\tnum},\hasType{x_\msf{acc}}{\tau}}{e_\msf{acc}}  {\tau} $

\textbf{ Induction \, Hypothesis } \text{:  For } $ e_\msf{arg}  ,\, e_\msf{base} , \, x_\msf{num}  ,\, x_\msf{acc} , \,e_\msf{acc}$ \text{ preservation rule holds true.} \\
\text{ Now we have 3 ways for } $\steps{ \rec{e_\msf{base}}{x_\msf{num}}{x_\msf{acc}}{e_\msf{acc}}{e_\msf{arg}}}{e^{`}}$

\textbf{(D-Rec-Step)} : Assume $\steps{e_\msf{arg}}{e^{`}_\msf{arg}}$
	\, so \,  $\steps{\rec{e_\msf{base}}{x_\msf{num}}{x_\msf{acc}}{e_\msf{acc}}{e_\msf{arg}}}{\rec{e_\msf{base}}{x_\msf{num}}{x_\msf{acc}}{e_\msf{acc}}{e^{`}_\msf{arg}}} $

\text{Now due to Induction Hypothesis we can have } $ e^{`}_\msf{arg} : \tau$
\text{and given premises  }\\
\text{by T-Rec} $\rec{e_\msf{base}}{x_\msf{num}}{x_\msf{acc}}{e_\msf{acc}}{e^{`}_\msf{arg}} : \tau$

\textbf{(D-Rec-Base)} : Assume \, $e_\msf{arg} = 0 \, \text{then by Inversion T-Num } e_\msf{arg} : \tnum \, so\, \\
 \rec{e_\msf{base}}{x_\msf{num}}{x_\msf{acc}}{e_\msf{acc}}{0} : \tau$
and $\steps{\rec{e_\msf{base}}{x_\msf{num}}{x_\msf{acc}}{e_\msf{acc}}{0}}{e_\msf{base}} $
\text{, now based on premises} $e_\msf{base} : \tau $

\textbf{(D-Rec-Dec)} : Assume $e_\msf{arg}$  =n and n  > 0

$\steps{\rec{e_\msf{base}}{x_\msf{num}}{x_\msf{acc}}{e_\msf{acc}}{n}}{[x_\msf{num} \rightarrow n, x_\msf{acc} \rightarrow \rec{e_\msf{base}}{x_\msf{num}}{x_\msf{acc}}{e_\msf{acc}}{n-1}] e_\msf{acc}}$

\text{now if} n : $\tnum $ \text{then for} (n - 1) \text{by inversion of T-Binop ,} (n-1) : $\number$
\text{, based on induction hypothesis we get} $\steps{x_\msf{num}}{n} \text{ and } \steps{x_\msf{acc}}{\rec{e_\msf{base}}{x_\msf{num}}{x_\msf{acc}}{e_\msf{acc}}{n-1}} $\text{ are type preserving operations, and from premises } $\typeJ{\ctx,\hasType{x_\msf{num}}{\tnum},\hasType{x_\msf{acc}}{\tau}}{e_\msf{acc}} {\tau}$

\text{then by the substitution typing lemma ( as reffered in lacture notes (foot note: 6)) }

$[x_\msf{num} \rightarrow n, x_\msf{acc} \rightarrow \rec{e_\msf{base}}{x_\msf{num}}{x_\msf{acc}}{e_\msf{acc}}{n-1}] \ e_\msf{acc} : \tau$

\text{Hence, preservation holds in either case.}

\end{proof}

\begin{proof}
Progress: if$\,  \emptyset \vdash e \, : \tau  \, \text{then} \, either\,  e\, \msf{val}\, or \, \steps{e}{e^\msf{`}}  . $\\
Proof : By rule induction on static semantics.\\

T-Rec: if $e = \rec{e_\msf{base}}{x_\msf{num}}{x_\msf{acc}}{e_\msf{acc}}{e_\msf{arg}} : \tau$ then either $\steps{e \, \msf{val}$ or\\
                     $\rec{e_\msf{base}}{x_\msf{num}}{x_\msf{acc}}{e_\msf{acc}}{e_\msf{arg}}}{e^\msf{`}}$

From premises we know that\\
$e_\msf{arg} : \tnum , e_\msf{base} : \tau , \, \,\, \typeJ{\hasType{x_\msf{num}}{\tnum},\hasType{x_\msf{acc}}{\tau}}{e_\msf{acc}}  {\tau} $

By the inductive hypothesis (IH), we got to assume that progress holds true for $e_\msf{arg}$ so either $e_\msf{arg} val$ or $\steps{e_\msf{arg}}{e^{`}_\msf{arg}}$

Now we case on different possible states of $e_\msf{arg}$ derived from IH:

A:  $\steps{e_\msf{arg}}{e^{`}_\msf{arg}}$ then by D-Rec-Step
 $\steps{\rec{e_\msf{base}}{x_\msf{num}}{x_\msf{acc}}{e_\msf{acc}}{e_\msf{arg}}}{\rec{e_\msf{base}}{x_\msf{num}}{x_\msf{acc}}{e_\msf{acc}}{e^{`}_\msf{arg}}} $
 
B: $e_\msf{arg} val$ and by premise $e_\msf{arg} : \tnum$ then by inversion of D-Num we know that $e_\msf{arg} = n$
 Now if n = 0 then by D-Rec-Base
 
$\steps{\rec{e_\msf{base}}{x_\msf{num}}{x_\msf{acc}}{e_\msf{acc}}{0}}{e_\msf{base}} $

if n > 0 then by D-Rec-Dec

$\steps{\rec{e_\msf{base}}{x_\msf{num}}{x_\msf{acc}}{e_\msf{acc}}{n}}{[x_\msf{num} \rightarrow n, x_\msf{acc} \rightarrow \rec{e_\msf{base}}{x_\msf{num}}{x_\msf{acc}}{e_\msf{acc}}{n-1}] e_\msf{acc}}$\\
In each case expression steps, so progress holds.
\end{proof}

\end{document}
